\section{Fuzzy Proximity and Similarity Relations}



%We define basic notions about similarity relations following \cite{DBLP:journals/tcs/Sessa02,DBLP:journals/fss/IranzoR17}.
%
A binary \emph{fuzzy relation} on a set $S$ is a mapping from $S\times S$ to the real interval $[0,1]$. If $\cR$ is a fuzzy relation on $S$ and $\cut$ is a number $0<\cut \le 1$ (called \emph{cut value}), then the \emph{$\cut$-cut} of $\cR$ on $S$, denoted $\cR_\cut$, is an ordinary (crisp) relation on $S$ defined as
$\cR_\cut := \{(s_1,s_2) \mid \cR(s_1,s_2) \ge \cut \}$. 

%We define basic notions about similarity relations following \cite{DBLP:journals/tcs/Sessa02,DBLP:journals/fss/IranzoR17}.

A fuzzy relation $\cR$ on a set $S$ is called a \emph{proximity relation}, if it is reflexive and symmetric:

\begin{description}
	\item[Reflexivity:] $\cR(s,s)=1$ for all $s\in S$;
	\item[Symmetry:] $\cR(s_1,s_2)=\cR(s_2,s_1)$ for all $s_1,s_2\in S$.
\end{description}

Let  $\wedge$ be a T-norm: an associative, commutative, non-decreasing binary operation on $[0,1]$ with 1 as the unit element. A proximity relation (on $S$) is called a \emph{similarity relation} (on $S$) iff it is transitive:

\begin{description}
	\item[Transitivity] $\cR(s_1,s_2)\ge \cR(s_1,s) \wedge \cR(s,s_2)$ for any $s_1,s_2,s\in S$.
\end{description}

In this paper, in the role of T-norm we take the \emph{minimum} of two numbers, and write $\min$ instead of $\wedge$. In the role of $S$ we take a syntactic domain, defined in the next section. 