\documentclass{article}
\bibliographystyle{plain}
\usepackage{bussproofs}
\usepackage{graphicx}%
\usepackage{amsmath,amssymb,amsfonts}%
\usepackage{amsthm}%
\usepackage{mathrsfs}%
\usepackage[title]{appendix}%
\usepackage{xcolor}%
\usepackage{textcomp}%
\usepackage{manyfoot}%
\usepackage{booktabs}%
\usepackage{algorithm}%
\usepackage{algorithmicx}%
\usepackage{algpseudocode}%
\usepackage{listings}%
\usepackage{stmaryrd}
\usepackage{hyperref}

%%%%
\usepackage{upgreek}
\usepackage{calc}
\usepackage{bm}

\newtheorem{theorem}{Theorem}%  meant for continuous numbers

\theoremstyle{proposition}
\newtheorem{proposition}{Proposition}% 
\newtheorem{counterexample}{Counterexample}
\newtheorem{lemma}{Lemma}
\newtheorem*{question}{Questions}
\newtheorem{conjecture}{Conjecture}
\newtheorem{corollary}{Corollary}

\theoremstyle{definition}%
\newtheorem{definition}{Definition}%
\newtheorem{example}{Example}%
\newtheorem{remark}{Remark}%

\def\univ{U}
\def\type{\mathit{type}}

\def\T{\mathcal{T}}

\def\P{\mathcal{P}}

\def\dom{\mathcal{D}}

\def\mod{\mathcal{M}}

\def\typetau{\uptau}
\def\typesigma{\upsigma}
\def\typevar{\upalpha}
\def\typebase{\upbeta}
\def\typeuniver{\upomega}
\newcommand{\tp}{\mathsf{type}}
\newcommand{\degree}{\mathsf{degree}}
\newcommand{\coltp}{\mathsf{ctype}}
\newcommand{\coldeg}{\mathsf{cdegree}}
\newcommand{\crisp}{\mathsf{crisp}}
\newcommand {\ie}{{\em i.e.}}
\newcommand {\eg}{{\em e.g.}}

\newcommand{\fit}{{\textsc{fA}}}
\newcommand{\fv}{\mathsf{fv}}
\newcommand{\bv}{\mathsf{bv}}

\newcommand\ddfrac[2]{\frac{\displaystyle #1}{\displaystyle #2}}


\newcommand{\temur}[1]{{\color{red}#1}}


\title{Notes on similarity and fuzzy membership}
\author{BD FH MK DM }
\begin{document}
\maketitle

\section{Similarity vs Fuzzy Memebrship}
There appear to be two approaches to {\em fuzzy regular languages}, the first uses the notion of {\em similarity} between elements of the alphabet and extends it thereof. The second uses the notion of ``{\em fuzzy membership}" of characters to alphabet, of words to languages, and generalizes it thereof. 
Similarity can be related to fuzzy membership if we have a binary equality relation, between characters. Namely if $(a,b)\in_\mu Eq$, then we can assume that ${\mathcal R}(a,b)=\mu$. Axioms must be added on $Eq$ membership for the induced relation to be a similarity. 

\section{Similarity}

Let ${\mathcal R}\subseteq D$ be a {\em similarity} on the domain (alphabet) $D$.

\begin{definition}\label{ext}Let ${\mathcal R}\subseteq D$ be a {\em similarity} on the domain (alphabet) $D$. ${\mathcal R}$ can be extended to \begin{itemize}
\item a {\em  relation on words} as follows:
 

${\mathcal R}^w(w_1,w_2) = \begin{cases} 0 & \mbox{ if } w_1= \epsilon \mbox{ and } w_2\neq \epsilon \\
{\mathcal R}(a_1,a_2)\otimes {\mathcal R}^w(w'_1,w'_2) & \mbox{ if }w_i=a_iw'_i \mbox{ and  } w'_1, w'_2\neq \epsilon\\
{\mathcal R}(a_1,a_2)\otimes K & \mbox{ if }w_i=a_iw'_i \mbox{ and  } w'_1= \epsilon \mbox{ or } w'_2 = \epsilon\\
\end{cases}$
\noindent The constant $K$  in the last clause, is some conventional value. This clause reflects the intuition behind the {\em lexicographic similarity} \ie\ the similarity one uses in looking up a word in the dictionary. 
\footnote{$$\frac{\mathcal{R}(a_1,a_2)\otimes\mathcal{R}(w_1',w_2')}{max(|w_1|,|w_2|)}  \mbox{ if }|w_1|\neq |w_2|
$$}
\item a {\em  relation on languages} as follows:
$${\mathcal R}^L(L_1,L_2)= \min\{\inf_{w_1\in L_1}\sup_{w_2\in L_2}{\mathcal R}^w(w_1,w_2),\inf_{w_2\in L_2}\sup_{w_1\in L_1}{\mathcal R}^w(w_1,w_2)\}.$$
\end{itemize}
\end{definition}



\begin{proposition}The relations ${\mathcal R}^w$ and ${\mathcal R}^L$ defined in Definition~\ref{ext} are similarity relations.
\end{proposition}

\begin{definition}\label{symexp} Let $e_1,e_2\subseteq A^*$ be regular expressions and  ${\mathcal R}$ be a similarity relation on $A$. We define a relation on regular expressions as follows:
\begin{itemize}
\item ${\mathcal R}^e(0,e) = \begin{cases} 1 & \mbox{ if } e=\emptyset\\
                                                        0 & \mbox{ otherwise };\end{cases}$
\item ${\mathcal R}^e(\epsilon,e) = \begin{cases} 1 \mbox{ if } \epsilon = e\\
0 & \mbox{ otherwise };\end{cases}$
\item ${\mathcal R}^e(a_1,a_2) = {\mathcal R}(a_1,a_2)$;
\item ${\mathcal R}^e(e_1^*,e_2^*) = {\mathcal R}^e(e_1,e_2)$;
\item ${\mathcal R}^e(e,f_1+f_2) = \min \{{\mathcal R}^e(e,f_1),{\mathcal R}^e(e,f_2)\}$;
\item ${\mathcal R}^e(e,f_1\wedge f_2) =\begin{cases} 1 &  e=\emptyset \mbox{ and } f_1\wedge f_2=\emptyset\\ \min \{{\mathcal R}^e(e,f_1),{\mathcal R}^e(e,f_2)\}& \mbox{ otherwise}\end{cases}$;
%\item ${\mathcal R}^e(e_1+e_2,f_1+f_2) = \min \{{\mathcal R}^e(e_1,f_1),{\mathcal R}^e(e_1,f_2),{\mathcal R}^e(e_2,f_1),{\mathcal R}^e(e_2,f_2) \}$;
%\item ${\mathcal R}^e(e_1\wedge e_2,f_1\wedge f_2) = \begin{cases} 1 & \mbox{ if } e_1\& e_2=\emptyset \mbox{ or } f_1\wedge f_2=\emptyset\\
%\min \{\max \{{\mathcal R}^e(e_1,f_1),{\mathcal R}^e(e_1,f_2)\}, &\\
%\ \ \ \ \ \ \ \ \ \ \max\{{\mathcal R}^e(e_2,f_1),{\mathcal R}^e(e_2,f_2) \}\} & \mbox{ otherwise}.
%\end{cases}$
\end{itemize}
\end{definition}
\noindent The last clause is more elaborate because we have to take care of the cases in which some  $e$ may be equal to some $f_i$.

\begin{conjecture}  The relation ${\mathcal R}^e$ defined in Definition~\ref{symexp} are similarity relations and moreover ${\mathcal R}^e(e_1,e_2)={\mathcal R}^L(L(e_1),L(e_2)).$
\end{conjecture}



\end{document}