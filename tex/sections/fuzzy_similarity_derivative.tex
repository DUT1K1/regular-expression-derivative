% !TEX root =main.tex
\section{Derivative of Regular Expressions with Fuzzy Similarity Relation}\label{sec:fuzzy-derivatives}
In this paper, we generalize this notion using fuzzy similarity. Suppose $\cR$ is a fuzzy similarity relation on $\Sigma$, and let $\cut$ be a cut value ($0 < \cut \le 1$). Then the \emph{$\cut$-fuzzy derivative} of $\mathcal{L}$ with respect to a symbol $a \in \Sigma$, denoted $D_a^\cut(\mathcal{L})$, is defined as the set of suffixes $w$ such that there exists $b \in \Sigma$ with $\cR(a,b) \ge \cut$ and $bw \in \mathcal{L}$. In this way, the derivative is taken with respect to all symbols $b$ sufficiently similar to $a$, with the leading $b$ removed.

\begin{definition}\label{fuzzyderivative}
$$
\partial_a^\cut(\mathcal{L}) = \{w \in \Sigma^* \mid \exists b \in \Sigma\ \text{such that}\ \cR(a,b) \ge \cut\ \text{and}\ bw \in \mathcal{L}\}.
$$
Alternatively:
$$
\partial_a^\cut(\mathcal{L}) = \{w \in \Sigma^* \mid \exists b \in \Sigma\ \text{such that}\ (a, b) \in \cR_\cut\ \text{and}\ bw \in \mathcal{L}\}
$$
\end{definition}

\noindent
Just as in the classical setting, the notion of a fuzzy derivative with respect to a single symbol can be extended to handle derivatives with respect to an entire word. The idea remains the same: process the word one symbol at a time from left to right, taking the fuzzy derivative with respect to the first symbol and then applying the fuzzy derivative again to the result for the remaining suffix. This recursive approach naturally generalizes the single-symbol fuzzy derivative to arbitrary words, as formalized below.


\begin{definition}
    The fuzzy derivative of a language $\mathcal{L} \subseteq \Sigma^*$ with respect to a word $u=a\cdot v \in \Sigma^*$ where $a \in \Sigma$ and $v \in \Sigma^*$ is defined to be:
    $$
    \partial^\cut_u(\mathcal{L}) = \partial^\cut_v(\partial^\cut_a(\mathcal{L}))
    $$
\end{definition}

\noindent
An important property of the classical derivative is that it preserves regularity. This ensures that repeated differentiation never produces a language outside the regular class, making the operation suitable for automata-based  algorithms. The same closure property holds in the fuzzy setting: even when derivatives are taken with respect to a cut-based similarity relation, the resulting language remains regular. The following theorem states this formally.


\begin{theorem}\label{the:language-fuzzy-derivative-regular}
    If $\mathcal{L} \subseteq \Sigma^*$ is regular, then 
    $\partial^\cut_u(\mathcal{L})$ is also regular for any word 
    $u \in \Sigma^*$ and for any cut value $\cut \in [0,1]$.
\end{theorem}

\begin{proof}
The argument is similar to the proof of Theorem~\ref{the:language-derivative-regular}.
Let $\mathcal{L} \subseteq \Sigma^*$ be regular and fix $\cut \in [0,1]$.

For a single symbol $a \in \Sigma$ we have, by Definition of $\partial_a^\cut$,
\[
  \partial_a^\cut(\mathcal{L})
  = \{ w \in \Sigma^* \mid \exists b \in \Sigma \text{ with } \cR(a,b) \ge \cut
       \text{ and } bw \in \mathcal{L} \}
  = \bigcup_{\substack{b \in \Sigma \\ \cR(a,b) \ge \cut}} \partial_b(\mathcal{L}),
\]
where $\partial_b(\mathcal{L})$ is the classical derivative of $\mathcal{L}$ with
respect to $b$. By Theorem~\ref{the:language-derivative-regular}, each
$\partial_b(\mathcal{L})$ is regular, and since $\Sigma$ is finite, the union
is finite; hence $\partial_a^\cut(\mathcal{L})$ is regular.

For a word $u = a_1 \cdots a_n$ we use the recursive definition
$\partial_u^\cut(\mathcal{L}) = \partial_{a_n}^\cut(\cdots \partial_{a_1}^\cut(\mathcal{L}) \cdots)$
and apply the previous argument inductively: at each step we take a fuzzy
derivative of a regular language and obtain a regular language again. Thus
$\partial_u^\cut(\mathcal{L})$ is regular for every $u \in \Sigma^*$ and every
cut $\cut \in [0,1]$.
\end{proof}


\noindent
In order to compute fuzzy derivatives in practice, it is convenient to identify, for a given symbol and cut value, the set of symbols that are considered similar enough to be treated as matches. We call this set the \emph{cut neighborhood} of the symbol, and define it formally as follows.


\begin{definition}\label{def:cut-neighborhood}
    $\text{For } a\in\Sigma \text{ and } \cut\in[0:1] \text{ we define cut neighborhood as}$
    $$
        \mathcal{S}^\cut_a = \{b \in\Sigma :  (a,b)\in\mathcal{R}_\cut\}
    $$
\end{definition}

\noindent
As in the classical case, we distinguish between the \emph{semantic} fuzzy derivative, defined directly on languages, and the \emph{syntactic} fuzzy derivative, defined on regular expressions. The syntactic formulation allows us to compute derivatives by manipulating the structure of the expression itself, without first converting it to its 
language. In the fuzzy setting, this requires taking into account all symbols in the cut neighborhood of the given symbol. The following definition formalizes the syntactic fuzzy derivative.


\begin{definition}\label{def:regex-fuzzy-derivative}
    Let $R$ be a regular expression over the alphabet $\Sigma$, let
    $a \in \Sigma$ be a symbol and let $\cut\in[0:1]$.
    The \emph{fuzzy derivative of the regular expression $R$ with respect to $a$ and $\cut$}
    is the regular expression
    $$
        \mathcal{D}^\cut_{a}(R)
    $$
    whose language satisfies
    $$
        \mathcal{L}\bigl(\mathcal{D}^\cut_a(R)\bigr) = \{v \in \Sigma^* : \exists b\in 
            \mathcal{S}^\cut_a \land b\cdot v \in \mathcal{L}(R)
        \}
    $$
\end{definition}

\noindent We define fuzzy derivative for regular expressions
$$
\begin{array}{c}
    \mathcal{D}^\cut_a(\varepsilon)=\emptyset \\[1ex]
   %\mathcal{D}^\cut_a(a)=\varepsilon \\[1ex]
 \mathcal{D}^\cut_a(b)=
      \begin{cases}
        \varepsilon & \text{if } b \in \mathcal{S}^\cut_a,\\
        \emptyset       & \text{otherwise.}
      \end{cases}\\[1ex]
    %\mathcal{D}^\cut_a(b)=\emptyset \text{ for } b \notin \mathcal{S}^\cut_a \\[1ex]
    \mathcal{D}^\cut_a(\emptyset)=\emptyset \\[1ex]
    \mathcal{D}^\cut_a(r\cdot s)= \mathcal{D}^\cut_a(r)\cdot s + \nu(r)\cdot \mathcal{D}^\cut_a(s) \\[1ex]
    \mathcal{D}^\cut_a(r^*)=\mathcal{D}^\cut_a(r)\cdot r^* \\[1ex]
    \mathcal{D}^\cut_a(r+s)=\mathcal{D}^\cut_a(r)+\mathcal{D}^\cut_a(s) \\[1ex]
    \mathcal{D}^\cut_a(r\:\&\:s)=\mathcal{D}^\cut_a(r)\:\&\:\mathcal{D}^\cut_a(s) \\[1ex]
    \mathcal{D}^\cut_a(\neg r) = \neg(\mathcal{D}^\cut_a(r))
\end{array}
$$

\noindent
The semantic and syntactic definitions of the fuzzy derivative are intended to capture the same intuitive operation: removing a given prefix (up to similarity) from all words in the language. To confirm this alignment, we state the following correspondence theorem, which ensures that taking the fuzzy derivative at the syntactic level and then interpreting it as a language yields exactly the same result as first interpreting the expression and then applying the semantic fuzzy derivative.


\begin{theorem}[Semantic-syntactic correspondence]\label{the:two-fuzzy-derivative-mapping}
    Let $R$ be a regular expression over $\Sigma$, $u\in\Sigma^*$ and $\cut\in[0:1]$. Then
    $$
        \mathcal{L}\bigr(\mathcal{D}^\cut_u(R)\bigl)=\partial^\cut_u\bigr(\mathcal{L}(R)\bigl)
    $$
\end{theorem}

\begin{proof}[Proof.]
We prove that for all regular expressions $R$ and all words $u \in \Sigma^*$,
\[
  \mathcal{L}\!\bigl(\mathcal{D}^\cut_u(R)\bigr)
  \;=\;
  \partial^\cut_u\!\bigl(\mathcal{L}(R)\bigr).
\]
As in the crisp case, the proof has two steps: first a single symbol $a\in\Sigma$,
then an arbitrary word $u$.

\paragraph{Step 1 (single symbol).}
Fix $a \in \Sigma$. We show by structural induction on $R$ that
\[
  \mathcal{L}\bigl(\mathcal{D}^\cut_a(R)\bigr)
  = \partial^\cut_a\bigl(\mathcal{L}(R)\bigr).
\]

\smallskip
\noindent\emph{Bases.}
\begin{itemize}
  \item $R = \emptyset$:
    $\mathcal{D}^\cut_a(\emptyset)=\emptyset$, hence
    $\mathcal{L}(\mathcal{D}^\cut_a(\emptyset))=\emptyset$, and
    $\partial^\cut_a(\emptyset)=\emptyset$ by definition.
  \item $R = \varepsilon$:
    $\mathcal{D}^\cut_a(\varepsilon)=\emptyset$, so
    $\mathcal{L}(\mathcal{D}^\cut_a(\varepsilon))=\emptyset$; on the language side
    there is no word of the form $bw$ (with $(a,b)\in\cR_\cut$) equal to $\varepsilon$,
    hence $\partial^\cut_a(\{\varepsilon\})=\emptyset$.
  \item $R = b \in \Sigma$:
    By Definition~\ref{def:regex-fuzzy-derivative},
    $\mathcal{D}^\cut_a(b)=\varepsilon$ if $b \in \mathcal{S}^\cut_a$ and
    $\mathcal{D}^\cut_a(b)=\emptyset$ otherwise, so
    \[
      \mathcal{L}(\mathcal{D}^\cut_a(b)) =
      \begin{cases}
        \{\varepsilon\} & \text{if } b \in \mathcal{S}^\cut_a,\\
        \emptyset       & \text{otherwise.}
      \end{cases}
    \]
    On the language side we have
    \[
      \partial^\cut_a(\{b\})
      = \{ w \mid \exists c\in\Sigma,\ (a,c)\in\cR_\cut \ \text{and}\ cw=b \},
    \]
    which is $\{\varepsilon\}$ exactly when $(a,b)\in\cR_\cut$ (i.e.\ $b\in\mathcal{S}^\cut_a$),
    and $\emptyset$ otherwise. Thus the two coincide.
\end{itemize}

\smallskip
\noindent\emph{Inductive cases.}
For $R = R_1 + R_2$, $R = R_1 \,\&\, R_2$, $R = \neg R_1$, $R = R_1\cdot R_2$ and
$R = (R_1)^*$, the syntactic fuzzy derivative is defined by exactly the same
algebraic clauses as in the crisp case (the parameter $\cut$ does not appear
in these rules), and the semantic fuzzy derivative $\partial^\cut_a$ is defined
as a finite union of classical derivatives:
\[
  \partial^\cut_a(L)
  = \bigcup_{\substack{b \in \Sigma \\ (a,b)\in\cR_\cut}} \partial_b(L).
\]
Since each classical derivative $\partial_b$ satisfies the usual algebraic
laws for $+$, $\&$, $\neg$, concatenation, and star, and finite unions preserve
these equalities, the calculations in the crisp proof carry over verbatim with
$\mathcal{D}_a$ and $\partial_a$ replaced by $\mathcal{D}^\cut_a$ and
$\partial^\cut_a$. Using the induction hypothesis on $R_1$ and $R_2$ in each
case, we obtain
\[
  \mathcal{L}\bigl(\mathcal{D}^\cut_a(R)\bigr)
  = \partial^\cut_a\bigl(\mathcal{L}(R)\bigr)
\]
for all $a \in \Sigma$ and all regular expressions $R$.

\paragraph{Step 2 (arbitrary word).}
We prove by induction on $u \in \Sigma^*$ that
\[
  \mathcal{L}\bigl(\mathcal{D}^\cut_u(R)\bigr)
  = \partial^\cut_u\bigl(\mathcal{L}(R)\bigr).
\]
\begin{itemize}
  \item Base $u=\varepsilon$:
    $\mathcal{D}^\cut_\varepsilon(R)=R$ and
    $\partial^\cut_\varepsilon(\mathcal{L}(R))=\mathcal{L}(R)$.
  \item Step $u = va$ with $v \in \Sigma^*$ and $a \in \Sigma$:
    \[
      \begin{aligned}
      \mathcal{L}\bigl(\mathcal{D}^\cut_{va}(R)\bigr)
      &= \mathcal{L}\bigl(\mathcal{D}^\cut_a(\mathcal{D}^\cut_v(R))\bigr) \\
      &= \partial^\cut_a\bigl(\mathcal{L}(\mathcal{D}^\cut_v(R))\bigr)
           \quad\text{(by Step 1)} \\
      &= \partial^\cut_a\bigl(\partial^\cut_v(\mathcal{L}(R))\bigr)
           \quad\text{(IH on $v$)} \\
      &= \partial^\cut_{va}\bigl(\mathcal{L}(R)\bigr).
      \end{aligned}
    \]
\end{itemize}
Therefore $\mathcal{L}(\mathcal{D}^\cut_u(R))=\partial^\cut_u(\mathcal{L}(R))$
for all words $u \in \Sigma^*$, which completes the proof.
\end{proof}


\subsection*{Examples}
Assume the alphabet $\Sigma = \{a, b, c\}$ and a fuzzy similarity relation $\cR$ on $\Sigma$ defined as follows:
\[
\begin{array}{c|ccc}
\cR & a & b & c \\
\hline
a & 1 & 0.8 & 0.4 \\
b & 0.8 & 1 & 0.5 \\
c & 0.4 & 0.5 & 1 \\
\end{array}
\]
Let the cut value be $\cut = 0.7$, so the $\cut$-cut of $\cR$ includes the pairs:
\[
\cR_\cut = \{(a, a), (a, b), (b, a), (b, b), (c, c)\}
\]
We can also observe that:
\[
\begin{array}{c}
     \mathcal{S}_a^\cut=\{a, b\} \\ [1ex]
     \mathcal{S}_b^\cut=\{a, b\} \\ [1ex]
     \mathcal{S}_a^\cut=\{c\}
\end{array}
\]
\noindent
Let $\mathcal{L} = \{abc, ba, bb\}$. Then:
\begin{itemize}
    \item $\partial_a^\cut(\mathcal{L}) = \{bc, a, b\}$\\
    Because:
    \begin{itemize}
        \item $(a,a) \in \cR_\cut$ and $abc \in \mathcal{L}$ $\Rightarrow$ $bc \in \partial_a^\cut(\mathcal{L})$
        \item $(a,b) \in \cR_\cut$ and $ba \in \mathcal{L}$ $\Rightarrow$ $a \in \partial_a^\cut(\mathcal{L})$
        \item $(a,b) \in \cR_\cut$ and $bb \in \mathcal{L}$ $\Rightarrow$ $b \in \partial_a^\cut(\mathcal{L})$
    \end{itemize}

    \item $\partial_c^\cut(\mathcal{L}) = \emptyset$\\
    Because:
    \begin{itemize}
        \item Only $(c,c) \in \cR_\cut$ with $\cR(c,c) = 1 \ge \cut$
        \item But $cb, ca, cc \notin \mathcal{L}$ $\Rightarrow$ no valid $bw$ with $(c,b) \in \cR_\cut$ and $bw \in \mathcal{L}$
    \end{itemize}

    \item $\partial_b^\cut(\mathcal{L}) = \{bc, a, b\}$\\
    Because:
    \begin{itemize}
        \item $(b,a) \in \cR_\cut$ and $abc \in \mathcal{L}$ $\Rightarrow$ $bc \in \partial_b^\cut(\mathcal{L})$
        \item $(b,b) \in \cR_\cut$ and $ba \in \mathcal{L}$ $\Rightarrow$ $a \in \partial_b^\cut(\mathcal{L})$
        \item $(b,b) \in \cR_\cut$ and $bb \in \mathcal{L}$ $\Rightarrow$ $b \in \partial_b^\cut(\mathcal{L})$
    \end{itemize}
\end{itemize}

\begin{theorem}\label{the:fuzzy-derivative-set-finite}
Let $\Sigma$ be a finite alphabet and let $\mathcal{L} \subseteq \Sigma^*$ be a
regular language. For every cut value $\cut \in [0,1]$, the set of fuzzy
derivatives
\[
  \bigl\{\, \partial_u^\cut(\mathcal{L}) \;\bigm|\; u \in \Sigma^* \bigr\}
\]
is finite.
\end{theorem}

\begin{proof}
Fix $\cut \in [0,1]$ and let $\mathcal{L}$ be regular. By
Theorem~\ref{the:derivative-set-is-finite}, the set of classical derivatives
\[
  D \;=\; \bigl\{\, \partial_w(\mathcal{L}) \mid w \in \Sigma^* \bigr\}
\]
is finite.

For a single symbol $a \in \Sigma$ we have, by the definition of the fuzzy
derivative,
\[
  \partial_a^\cut(\mathcal{L})
  = \{ w \mid \exists b \in \Sigma \text{ with } \cR(a,b) \ge \cut
                  \text{ and } bw \in \mathcal{L} \}
  = \bigcup_{\substack{b \in \Sigma \\ \cR(a,b) \ge \cut}} \partial_b(\mathcal{L}),
\]
so $\partial_a^\cut(\mathcal{L})$ is a finite union of elements of $D$.

Using the recursive definition
$\partial_{av}^\cut(\mathcal{L}) = \partial_v^\cut(\partial_a^\cut(\mathcal{L}))$
and the fact that $\partial_a^\cut$ distributes over unions, a straightforward
induction on the length of $u$ shows that every fuzzy derivative
$\partial_u^\cut(\mathcal{L})$ is a finite union of languages from $D$.
Since $D$ is finite, there are only finitely many such unions (at most
$2^{|D|}$). Hence the set
$\{\partial_u^\cut(\mathcal{L}) \mid u \in \Sigma^*\}$ is finite.
\end{proof}



\begin{theorem}\label{the:regex-fuzzy-derivative-types-finite}
Let $\Sigma$ be a finite alphabet and let $R$ be a regular expression over
$\Sigma$. For every cut value $\cut \in [0,1]$, the set of derivative languages
\[
  \bigl\{\, \mathcal{L}(\mathcal{D}^\cut_{w}(R)) \;\bigm|\; w \in \Sigma^* \bigr\}
\]
is finite.
\end{theorem}

\begin{proof}
Let $\cut \in [0,1]$ be fixed and $\mathcal{L} = \mathcal{L}(R)$.
By Theorem~\ref{the:two-fuzzy-derivative-mapping}, for every word $w \in \Sigma^*$,
\[
  \mathcal{L}\bigl(\mathcal{D}^\cut_w(R)\bigr)
  \;=\;
  \partial^\cut_w\bigl(\mathcal{L}(R)\bigr)
  \;=\;
  \partial^\cut_w(\mathcal{L}).
\]
Hence
\[
  \bigl\{\, \mathcal{L}(\mathcal{D}^\cut_{w}(R)) \mid w \in \Sigma^* \bigr\}
  \;=\;
  \bigl\{\, \partial^\cut_w(\mathcal{L}) \mid w \in \Sigma^* \bigr\},
\]
and the right-hand set is finite by
Theorem~\ref{the:fuzzy-derivative-set-finite}. This proves the claim.
\end{proof}



\begin{theorem}\label{the:fuzzy-dfa-correctness}
Let $r$ be a regular expression over $\Sigma$, let $\cut \in [0,1]$, and let
$M^\cut = \langle Q^\cut, \Sigma, \delta^\cut, q_0^\cut, F^\cut \rangle$
be the DFA constructed from $r$ using fuzzy derivatives (for the fixed cut
value $\cut$).
Then $M^\cut$ and $r$ define the same $\cut$-language:
\[
  \mathcal{L}(M^\cut) \;=\; \mathcal{L}^\cut(r).
\]
Equivalently, for every word $w \in \Sigma^*$,
\[
  M^\cut \text{ accepts } w
  \;\;\Longleftrightarrow\;\;
  w \in \mathcal{L}^\cut(r).
\]
\end{theorem}

\begin{proof}[Proof.]
Let $\delta^{\cut *} : Q^\cut \times \Sigma^* \to Q^\cut$ be the standard
extension of $\delta^\cut$ to words:
\[
  \delta^{\cut *}(q,\varepsilon) = q,
  \qquad
  \delta^{\cut *}(q,wa) = \delta^\cut\bigl(\delta^{\cut *}(q,w),a\bigr)
  \text{ for } w \in \Sigma^*, a \in \Sigma.
\]

We first show by induction on $w \in \Sigma^*$ that
\[
  \delta^{\cut *}(q_0^\cut,w) = \bigl[\,\mathcal{D}^\cut_w(r)\,\bigr].
\]

\emph{Base case:} For $w = \varepsilon$ we have
$q_0^\cut = [r] = [\mathcal{D}^\cut_\varepsilon(r)]$, so the claim holds.

\emph{Inductive step:} Let $w = ua$ with $u \in \Sigma^*$ and $a \in \Sigma$.
By the induction hypothesis,
$\delta^{\cut *}(q_0^\cut,u) = [\mathcal{D}^\cut_u(r)]$.
Then, by the definition of $\delta^\cut$,
\[
  \delta^{\cut *}(q_0^\cut,ua)
  = \delta^\cut\bigl(\delta^{\cut *}(q_0^\cut,u),a\bigr)
  = \delta^\cut\bigl([\mathcal{D}^\cut_u(r)],a\bigr)
  = \bigl[\,\mathcal{D}^\cut_a(\mathcal{D}^\cut_u(r))\,\bigr]
  = \bigl[\,\mathcal{D}^\cut_{ua}(r)\,\bigr],
\]
using the rule $\mathcal{D}^\cut_{ua}(r) = \mathcal{D}^\cut_a(\mathcal{D}^\cut_u(r))$.
This proves the claim.

Now let $w \in \Sigma^*$. Then:
\[
\begin{aligned}
w \text{ is accepted by } M^\cut
&\Longleftrightarrow \delta^{\cut *}(q_0^\cut,w) \in F^\cut \\[0.5ex]
&\Longleftrightarrow \delta^{\cut *}(q_0^\cut,w) = [\mathcal{D}^\cut_w(r)]
             \text{ and } [\mathcal{D}^\cut_w(r)] \in F^\cut
             \quad\text{(by the claim above and def.\ of $F^\cut$)} \\[0.5ex]
&\Longleftrightarrow \nu\bigl(\mathcal{D}^\cut_w(r)\bigr) = \varepsilon \\[0.5ex]
&\Longleftrightarrow \varepsilon \in \mathcal{L}\bigl(\mathcal{D}^\cut_w(r)\bigr) \\[0.5ex]
&\Longleftrightarrow \varepsilon \in \partial^\cut_w\bigl(\mathcal{L}(r)\bigr)
     \quad\text{(by Theorem~\ref{the:two-fuzzy-derivative-mapping})} \\[0.5ex]
&\Longleftrightarrow w \in \mathcal{L}^\cut(r)
     \quad\text{(by the definition of $\mathcal{L}^\cut(r)$).}
\end{aligned}
\]
Hence $\mathcal{L}(M^\cut) = \mathcal{L}^\cut(r)$, as required.
\end{proof}
