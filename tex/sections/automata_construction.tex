\section{Automata Construction}\label{sec:dfa-from-derivatives}

In this section we construct a finite deterministic automaton (DFA) from a regular expression $r$, using the derivative operator $\mathcal{D}_a(\cdot)$
and the nullable function $\nu(\cdot)$. Each DFA state denotes the remaining
language of $r$ after consuming the prefix. Transitions are computed by taking one-step
derivatives with respect to input symbols.

\subsection{DFA Structure}
Let $\equiv$ denote semantic equivalence of regular expressions:
\[
  R \equiv S \iff \mathcal{L}(R)=\mathcal{L}(S).   
  \]
We define the DFA $M=\langle Q, \Sigma, \delta,  q_0, F\rangle,$ as follows.

\begin{itemize}
    \item $Q \;=\; \{\, [\,\mathcal{D}_w(r)\,] \mid w \in \Sigma^* \,\}$ is the set of states; 

    \item $\Sigma$ is a finite alphabet.

  \item $\delta: Q \times \Sigma \to Q$ is the transition function defined by
        \[
          \delta\bigl([R], a\bigr) \;=\; \bigl[\,\mathcal{D}_a(R)\,\bigr].
        \]
    \item $q_0 = [\,r\,] \in Q$ is the start state (since $\mathcal{D}_\varepsilon(r)=r$).
  \item $F \subseteq Q$ is the set of accepting states:
        \[
          F \;=\; \{\, [R] \in Q \mid \nu(R)=\varepsilon \,\}.
        \]
\end{itemize}



By Theorem~\ref{the:two-derivative-mapping},
$\mathcal{L}(\mathcal{D}_u(r))=\partial_u(\mathcal{L}(r))$, running the DFA corresponds exactly to repeatedly taking derivatives.


\subsection{Pseudocode}\label{subsec:pseudocode-deriv-dfa}


\begin{itemize}
  \item \texttt{Sigma} $\equiv \Sigma$.
  \item \texttt{delta} $\equiv \delta : Q\times\Sigma\to Q$.
  \item \texttt{State} $:= [R]$.
  \item \texttt{D(a, R)} $\equiv \mathcal{D}_a(R)$.
  \item \texttt{nullable(R)} $\equiv \nu(R)=\varepsilon$.
  \item \texttt{epsilon} $\equiv \varepsilon$.
  \item \texttt{q\_0} $\equiv [r]$ (since $\mathcal{D}_\varepsilon(r)=r$).
  \item \texttt{empty} $\equiv \emptyset$.
  \item \texttt{union}(Q,\{x\}) $\equiv Q \cup \{x\}$.
\end{itemize}



\begin{lstlisting}[style=thesiscode,label={lst:deriv-dfa-paper}]
 goto(q, a, Q, delta):
    q_a := [ D(a, q) ]
    if exists q' in Q and q' = q_a then
        delta[(q, a)] := q'
        return (Q, delta)
    else
        Q     := union(Q, { q_a })
        delta := union(delta, (q, a) -> q_a))
        for each a' in Sigma do
            (Q, delta) := goto(q_a, a', Q, delta)
        return (Q, delta)

dfa(r):
    q_0 := [ r ]
    Q  := { q_0 }
    delta := empty
    for each a in Sigma do
        (Q, delta) := goto(q_0, a, Q, delta)
    F := { q in Q | nullable(q) }
    return (Q, Sigma, delta, q_0, F)
\end{lstlisting}

\textbf{Termination.}
By Theorem~\ref{the:regex-derivative-types-finite}, only finitely many
distinct derivative languages arise; so the construction terminates.




\begin{theorem}\label{the:dfa-correctness}
Let $r$ be a regular expression over $\Sigma$, and let
$M = \langle Q, \Sigma, \delta, q_0, F \rangle$
be the DFA constructed from $r$.
Then $M$ and $r$ define the same language:
\[
  \mathcal{L}(M) \;=\; \mathcal{L}(r).
\]
Equivalently, for every word $w \in \Sigma^*$,
\[
  M \text{ accepts } w
  \;\;\Longleftrightarrow\;\;
  w \in \mathcal{L}(r).
\]
\end{theorem}

\begin{proof}
Let $\delta^* : Q \times \Sigma^* \to Q$ be the extension of
$\delta$ to words:
\[
  \delta^*(q,\varepsilon) = q,
  \qquad
  \delta^*(q,wa) = \delta\bigl(\delta^*(q,w),a\bigr)
  \text{ for } w \in \Sigma^*, a \in \Sigma.
\]

We first show by induction on $w \in \Sigma^*$ that
\[
  \delta^*(q_0,w) = \bigl[\,\mathcal{D}_w(r)\,\bigr].
\]
\emph{Base case:} For $w = \varepsilon$ we have
$q_0 = [r] = [\mathcal{D}_\varepsilon(r)]$, so the claim holds.

\emph{Inductive step:} Let $w = ua$ with $u \in \Sigma^*$ and $a \in \Sigma$.
By the induction hypothesis,
$\delta^*(q_0,u) = [\mathcal{D}_u(r)]$.
Then, by the definition of $\delta$,
\[
  \delta^*(q_0,ua)
  = \delta\bigl(\delta^*(q_0,u),a\bigr)
  = \delta\bigl([\mathcal{D}_u(r)],a\bigr)
  = \bigl[\,\mathcal{D}_a(\mathcal{D}_u(r))\,\bigr]
  = \bigl[\,\mathcal{D}_{ua}(r)\,\bigr],
\]
using the rule $\mathcal{D}_{ua}(r) = \mathcal{D}_a(\mathcal{D}_u(r))$.
This proves the claim.

Now let $w \in \Sigma^*$. Then:
\[
\begin{aligned}
w \text{ is accepted by } M
&\Longleftrightarrow \delta^*(q_0,w) \in F \\[0.5ex]
&\Longleftrightarrow \delta^*(q_0,w) = [\mathcal{D}_w(r)]
             \text{ and } [\mathcal{D}_w(r)] \in F
             \quad\\[0.5ex]
&\Longleftrightarrow \nu\bigl(\mathcal{D}_w(r)\bigr) = \varepsilon \\[0.5ex]
&\Longleftrightarrow \varepsilon \in \mathcal{L}\bigl(\mathcal{D}_w(r)\bigr) \\[0.5ex]
&\Longleftrightarrow \varepsilon \in \partial_w\bigl(\mathcal{L}(r)\bigr)
     \quad\text{(by Theorem~\ref{the:two-derivative-mapping})} \\[0.5ex]
&\Longleftrightarrow w\varepsilon \in \mathcal{L}(r)
     \quad\text{(by the definition of }\partial_w\text{)} \\[0.5ex]
&\Longleftrightarrow w \in \mathcal{L}(r)
     \quad\text{(since }w\varepsilon = w\text{).}
\end{aligned}
\]
Hence $\mathcal{L}(M) = \mathcal{L}(r)$.

\end{proof}