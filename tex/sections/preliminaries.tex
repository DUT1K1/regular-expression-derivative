\section{Preliminaries}

We start by providing foundational definitions necessary for understanding the rest of the paper.

\subsection{Alphabets and Languages}

An \emph{alphabet} $\Sigma$ is a finite nonempty set of symbols. A \emph{word} over $\Sigma$ is a finite sequence of symbols from $\Sigma$. The empty word is denoted by $\varepsilon$. We denote the set of all words over $\Sigma$ by $\Sigma^*$, and a \emph{language} over $\Sigma$ is a subset $\mathcal{L} \subseteq \Sigma^*$.

\subsection{Regular Expressions: Syntax}

Regular expressions over an alphabet $\Sigma$ are formally defined by the following grammar:

$$
R ::= \emptyset \mid \varepsilon \mid a\,(a \in \Sigma) \mid R+R \mid R\cdot R \mid R^* \mid R \& R \mid \neg R
$$

The intuitive meanings of these expressions are:
\begin{itemize}
    \item $\emptyset$ represents the empty language,
    \item $\varepsilon$ represents the language containing only the empty word,
    \item $a$ (for $a \in \Sigma$) represents the language containing only the word $a$,
    \item $R + S$ represents the union of the languages represented by $R$ and $S$,
    \item $R \cdot S$ represents the concatenation of languages represented by $R$ and $S$,
    \item $R^*$ represents the Kleene star (iterated concatenation) of the language represented by $R$,
    \item $R \& S$ represents the intersection of languages represented by $R$ and $S$,
    \item $\neg R$ represents the complement of the language represented by $R$.
\end{itemize}

\subsection{Semantic Interpretation}

Given a regular expression $R$ over $\Sigma$, we define its semantic interpretation as the language $\mathcal{L}(R)$ as follows:
\[
\begin{aligned}
\mathcal{L}(\emptyset) &= \emptyset \\
\mathcal{L}(\varepsilon) &= \{\varepsilon\} \\
\mathcal{L}(a) &= \{a\}, \quad a \in \Sigma \\
\mathcal{L}(R+S) &= \mathcal{L}(R) \cup \mathcal{L}(S) \\
\mathcal{L}(R\cdot S) &= \{xy \mid x \in \mathcal{L}(R), y \in \mathcal{L}(S)\} \\
\mathcal{L}(R^*) &= \bigcup_{n \geq 0} \mathcal{L}(R)^n, \quad \text{where } \mathcal{L}(R)^0 = \{\varepsilon\}, \quad \mathcal{L}(R)^{n+1} = \mathcal{L}(R)^n \cdot \mathcal{L}(R) \\
\mathcal{L}(R \& S) &= \mathcal{L}(R) \cap \mathcal{L}(S) \\
\mathcal{L}(\neg R) &= \Sigma^* \setminus \mathcal{L}(R)
\end{aligned}
\]

With these preliminaries, we have clearly set the foundations required to understand the derivative operations defined subsequently.

\subsection{Nullable Regular Languages}
In many constructions involving regular expressions, it is useful to know 
whether the language described by a given expression contains the empty word 
$\varepsilon$.  
This property, called \emph{nullability}, plays a crucial role in the 
definition of derivatives, particularly when handling concatenation: 
if the first part of a concatenation can generate $\varepsilon$, 
then the derivative must also account for the derivative of the second part.  
To formalize this property, we introduce the \emph{nullable} function $\nu$, 
which determines whether a regular expression is nullable and provides 
a convenient symbolic form for use in subsequent rules.

\begin{definition}\label{def:nullable}
    If a regular language contains $\varepsilon$ it is called nullable. $\varepsilon \in \mathcal{L}[\![r]\!]$. And we define the following function:
$$
\nu(r) = 
\begin{cases}
	\varepsilon & \text{if }r \text{ is nullable} \\
	\emptyset & \text{otherwise}
\end{cases}
$$
\end{definition}

\noindent
Now based on the definition~\ref{def:nullable} we can write the following:
$$
\begin{array}{c}
     \nu (\varepsilon)=\varepsilon  \\[1ex]
     \nu(a)=\emptyset \\[1ex]
     \nu(\emptyset)=\emptyset \\[1ex]
     \nu(r\cdot s)=\nu(r)\: \& \: \nu(s) \\[1ex]
     \nu(r+s) = \nu(r) + \nu(s) \\[1ex]
     \nu(r^*) = \varepsilon \\[1ex]
     \nu(r\:\&\:s) = \nu(r) \: \& \: \nu(s) \\[1ex]
     \nu(\neg r)= 
     \begin{cases}
         \varepsilon & \text{if } \nu(r) = \emptyset \\
         \emptyset & \text{if } \nu(r) = \varepsilon
     \end{cases}
\end{array}
$$


\subsection{Canonical Form}
 $[R]$ denotes the canonical (normalized) form
of the regular expression $R$.
\\
$[R]=[S] \implies$
$\mathcal{L}(R)=\mathcal{L}(S)$.
